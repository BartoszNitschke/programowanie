\documentclass[aspectratio=169]{beamer}

\usepackage[utf8]{inputenc}
\usepackage{textcomp}
\usepackage[polish]{babel}
\usepackage{amsthm}
\usepackage{graphicx}
\usepackage[T1]{fontenc}
\usepackage{scrextend}
\usepackage{hyperref}
\usepackage{xcolor}
\usepackage{geometry}
\usepackage{listings}
\usepackage{epigraph}

\renewcommand{\epigraphsize}{\scriptsize}

\usetheme{-bjeldbak/beamerthemebjeldbak}

\definecolor{xbarcolor}{HTML}{000000}
\setbeamercolor{lower separation line head}{bg=xbarcolor}
\setbeamercolor{lower separation line foot}{bg=xbarcolor}

\title{Podstawy programownia (w języku C++)}
\subtitle{I/O (operacje wejścia-wyjścia)}
\author{Marek Marecki}
\institute{Polsko-Japońska Akademia Technik Komputerowych}

\lstset{basicstyle=\ttfamily\color{black},
columns=fixed,
escapeinside={[*}{*]},
inputencoding=utf8,
extendedchars=true,
moredelim=**[is][\color{red}]{@}{@},
moredelim=**[is][\color{gray}]{`}{`},
moredelim=**[is][\color{olive}]{$}{$}}

\begin{document}

{%
    \setbeamertemplate{headline}{}
    \frame{\titlepage}
}

\section{Wstęp}

\begin{frame}
    \frametitle{I/O}

    I/O czyli ,,operacje wejścia wyjścia'' (ang. \emph{input-output}) pozwala
    programom na interakcję ze światem zewnętrznym.

    Bez I/O nie byłoby możliwe napisanie nawet tak prostego programu jak
    klasyczne ,,Hello,~World!''.
\end{frame}

\begin{frame}
    \frametitle{C++: I/O streams}

    Biblioteka standardowa języka C++ dostarcza tzw. ,,I/O streams'': struktury
    danych, które opakowują mechanizmy I/O dostarczane przez system operacyjny.
    Są one proste w użyciu (np. \texttt{std::cout}, \texttt{std::cin}) i oferują
    spójny interfejs ,,strumienia'' również dla typów, które udają I/O, np.
    \texttt{std::ostringstream} i \texttt{std::istringstream}.

    \vspace{1em}

    I/O streams jednak sprawdzają się średnio w sytuacjach kiedy potrzeba jest
    zrelizować coś co wykracza poza ich ,,happy path'', oraz nie oferują nic w
    zakresie I/O z wykorzystaniem sieci.

    Z tego powodu pominiemy je i zajmiemy się mechanizmami I/O oferowanymi przez
    system operaycjny Linux.
\end{frame}

\begin{frame}
    \frametitle{Linux: file descriptors}

    Linux jest systemem Unixopodobnym (ang. \emph{*nix-like}) i zapewnia
    mechanizmy I/O opisane przez standard POSIX.

    \vspace{1em}

    Oprócz mechanizmów I/O narzuconych przez standard POSIX, Linux oferuje swoje
    własne API (np. \texttt{io\_uring}), jednak w tym wykładzie nie będzie ono
    omawiane.
\end{frame}

\begin{frame}
    \frametitle{POSIX I/O: Everything is a file}

    Systemy POSIX działają zgodnie z zasadą ,,everything is a file'' czyli
    ,,wszystko jest plikiem''. Dzięki temu możemy używając jednego, spójnego
    interfejsu manipulować sporą ilością rzeczy w systemie.

    \vspace{1em}

    Pod kątem tego wykładu ważniejsze jednak będzie to, że ta zasada oznacza, że
    bardzo podobny kod pozwoli nam obsłużyć I/O na plikach jak i na innych
    metodach przenoszenia danych.
\end{frame}

\section{File I/O}

\begin{frame}
    \frametitle{API}

    \begin{enumerate}
        \item \texttt{open(2)}
        \item \texttt{read(2)}
        \item \texttt{write(2)}
        \item \texttt{lseek(2)}
        \item \texttt{stat(2)}, \texttt{lstat(2)}
        \item \texttt{close(2)}
    \end{enumerate}
\end{frame}

\begin{frame}[fragile]
    \frametitle{Otwieranie i zamykanie pliku}
    \framesubtitle{\texttt{open(2)}, \texttt{close(2)}}

    Stworzenie pliku i z prawami do zapisu i odczytu przez użytkownika, a
    następnie (po użyciu) zamknięcie file descriptora, który na niego wskazuje:

    {\footnotesize
    \begin{lstlisting}
    auto name = std::string{"./example.txt"};
    auto fd = @open@(name.c_str(), O_CREAT|O_RDWR, S_IRUSR|S_IWUSR);

    $// do something with fd$

    @close@(fd);
    \end{lstlisting}}

    Funkcja \texttt{open(2)} może zwrócić wartość -1 w przypadku
    błędu\footnote{Zdecydowana większość funkcji z API POSIX zwraca -1 w
    przypadku błędu, i ustawia globalną zmienną \texttt{errno} na wartość
    informującą o tym co dokładnie poszło nie tak.}!
\end{frame}

\begin{frame}[fragile]
    \frametitle{Zapis do deskryptora}
    \framesubtitle{\texttt{write(2)}}

    Zapis danych polega na podaniu funkcji \texttt{write(2)} wskaźnika do
    pewnego obszaru pamięci, oraz podaniu ilości bajtów, która ma być z tego
    obszaru zapisana do file descriptora:

    {\footnotesize
    \begin{lstlisting}
    auto buf = std::string{"Hello, World!\n"};
    auto n = @write@(fd, buf.data(), buf.size());
    if (n == -1) {
        perror("write(2)");
    }
    \end{lstlisting}}
\end{frame}

\begin{frame}[fragile]
    \frametitle{Odczyt z deskryptora}
    \framesubtitle{\texttt{read(2)}}

    Odczyt danych polega na podaniu funkcji \texttt{read(2)} wskaźnika do
    pewnego obszaru pamięci, oraz podaniu ilości bajtów, która powinna być do
    niego wczytana\footnote{należy pamiętać, żeby nie podać wartości większej
    niż rozmiar bufora}:

    {\footnotesize
    \begin{lstlisting}
    std::array<char, 4096> buf { 0 };
    auto const n = @read@(fd, buf.data(), buf.size());
    if (n == -1) {
        perror("read(2)");
    } else {
        std::cout << std::string{buf.data(), static_cast<size_t>(n)};
    }
    \end{lstlisting}}
\end{frame}

\begin{frame}
    \frametitle{Częściowy sukces}
    \framesubtitle{\texttt{write(2)}, \texttt{read(2)}}

    Dla funkcji \texttt{write(2)} i \texttt{read(2)} w przypadku sukcesu
    wynikiem jest ilość zapisanych lub odczytanych bajtów. Nie ma
    \emph{absolutnie żadnej gwarancji}, że ta ilość będzie dokładnie równa
    ilości przekazanej jako argument\footnote{ta sama uwaga dotyczy I/O
    wykonywanego po sieci}.

    \vspace{1em}

    To ile bajtów jest możliwe do zapisu lub odczytu nie zależy tylko od
    naszego programu. Mają na to wpływ inne uruchomione procesy, rozmiary
    buforów systemu operacyjnego, scheduler I/O, prędkość dysku i łącza, itd.
\end{frame}

\begin{frame}[fragile]
    \frametitle{Informacje o pliku}
    \framesubtitle{\texttt{fstat(2)}}

    Używając funkcji \texttt{fstat(2)} można odczytać np. rozmiar pliku, żeby
    móc określić ile bajtów potrzeba żeby wczytać jego całość:

    {\footnotesize
    \begin{lstlisting}
    struct stat info;
    memset(&info, 0, sizeof(info));
    auto const r = fstat(fd, &info);
    if (r == -1) {
        perror("stat(2)");
    } else {
        std::cout << info.st_size << " byte(s)\n";
    }
    \end{lstlisting}}
\end{frame}

\begin{frame}[fragile]
    \frametitle{Zadanie: zapis studenta}
    \label{lecture_exercise_0}

    Korzystając ze struktury danych \texttt{Student} z poprzednich zajęć napisz
    program, który zapisze ją do pliku \texttt{student.txt}. Niech każde pole
    będzie zapisane w osobnej linijce (tj. zakończone znakiem nowej linii).

    Kod źródłowy: \texttt{s06-store-student.cpp}

    \vspace{1em}

    Przykładowy plik:
    {\small
    \begin{lstlisting}
    Student Studencki
    s12345
    1
    4.25
    \end{lstlisting}}
\end{frame}

\begin{frame}
    \frametitle{Zadanie: odczyt studenta}
    \label{lecture_exercise_1}

    Korzystając ze struktury danych \texttt{Student} z poprzednich zajęć napisz
    program, który wczyta ją z pliku \texttt{student.txt} (niech wczyta to co
    zapisał program z poprzedniego zadania).

    Kod źródłowy: \texttt{s06-load-student.cpp}
\end{frame}

\begin{frame}
    \frametitle{Zadanie: drukowanie pliku}
    \label{lecture_exercise_2}

    Napisz program, który pobierze jako argument (\texttt{argv}) nazwę pliku,
    oraz wydrukuje zawartość tego pliku na standardowe wyjście. Obsłuż sytuację,
    w której plik nie istnieje.

    Kod źródłowy: \texttt{s06-cat.cpp}
\end{frame}

\section{Network I/O}

\begin{frame}
    \frametitle{API}

    \begin{enumerate}
        \item \texttt{socket(2)}
        \item \texttt{bind(2)}
        \item \texttt{listen(2)}
        \item \texttt{connect(2)}
        \item \texttt{read(2)}
        \item \texttt{write(2)}
        \item \texttt{shutdown(2)}
        \item \texttt{close(2)}
        \item \texttt{signal(2)} (blokada \texttt{SIGPIPE})
        \item \texttt{ip(7)}, \texttt{ipv6(7)}
        \item \texttt{tcp(7)}, \texttt{udp(7)}
    \end{enumerate}
\end{frame}

\begin{frame}[fragile]
    \frametitle{Utworzenie socketu TCP/IP}
    \framesubtitle{\texttt{socket(2)}}

    {\footnotesize
    \begin{lstlisting}
    auto sock = socket(AF_INET, SOCK_STREAM, 0);
    \end{lstlisting}}

    Stworzy to ,,goły'' socket TCP/IP, który można potem wykorzystać do
    nasłuchiwania na połączenia (server), albo do nawiązania połączenia ze
    zdalną maszyną (client).
\end{frame}

\begin{frame}[fragile]
    \frametitle{Zamknięcie połączenia}
    \framesubtitle{\texttt{shutdown(2)}}

    {\footnotesize
    \begin{lstlisting}
    shutdown(sock, SHUT_RDWR);
    \end{lstlisting}}

    Zamknie to kanały komunikacyjne i w poprawny sposób zakończy połączenie.
    Zasoby (file descriptor i socket) nie będą zwolnione do systemu
    operacyjnego.
\end{frame}

\begin{frame}[fragile]
    \frametitle{Zamknięcie socketu}
    \framesubtitle{\texttt{close(2)}}

    {\footnotesize
    \begin{lstlisting}
    close(sock);
    \end{lstlisting}}

    Zamknięcie socketu i file descriptora, oraz oddanie zasobów do systemu
    operaycjnego.
\end{frame}

\begin{frame}[fragile]
    \frametitle{Określenie adresu}
    \framesubtitle{\texttt{inet\_pton(2)}, \texttt{sockaddr\_in}}

    {\footnotesize
    \begin{lstlisting}
    auto const ip = std::string{"192.168.0.24"};
    auto const port = uint16_t{8080};

    sockaddr_in addr;
    memset(&addr, 0, sizeof(addr));
    addr.sin_family = AF_INET;
    addr.sin_port = htobe16(port);
    @inet_pton@(addr.sin_family, ip.c_str(), &addr.sin_addr);
    \end{lstlisting}}

    W przypadku serwera - na jakim adresie będzie nasłuchiwać na połączenia.
    W przypadku klienta - na jaki adres spróbuje nawiązać połączenie.
\end{frame}

\subsection{Server}

\begin{frame}[fragile]
    \frametitle{Przywiązanie do adresu}
    \framesubtitle{\texttt{bind(2)}}

    {\footnotesize
    \begin{lstlisting}
    sockaddr_in addr;
    $// prepare the address$

    @bind@(sock, reinterpret_cast<sockaddr*>(&addr), sizeof(addr));
    \end{lstlisting}}
\end{frame}

\begin{frame}[fragile]
    \frametitle{Nasłuchiwanie na połączenia}
    \framesubtitle{\texttt{listen(2)}}

    {\footnotesize
    \begin{lstlisting}
    @listen@(sock, SOMAXCONN); $// or 0 for implementation-default value$
    \end{lstlisting}}

    Argument podany jako \texttt{SOMAXCONN} określa maksymalną liczbę połączeń
    oczekujących na odebranie.
\end{frame}

\begin{frame}[fragile]
    \frametitle{Odebranie połączenia}
    \framesubtitle{\texttt{accept(2)}}

    {\footnotesize
    \begin{lstlisting}
    auto client_sock = @accept@(sock, nullptr, nullptr);
    \end{lstlisting}}

    Zmienna \texttt{client\_sock} zawiera file descriptor wskazujący na socket,
    który umożliwia komunikację z klientem.

    Dodatkowe dwa parametry wskazują na strukturę danych (i jej rozmiar), do
    której system operacyjny powinien zapisać adres klienta. Jeśli ta informacja
    nie jest potrzebna, można przekazać \texttt{nullptr}.
\end{frame}

\subsection{Client}

\begin{frame}[fragile]
    \frametitle{Nawiązanie połączenia}
    \framesubtitle{\texttt{connect(2)}}

    {\footnotesize
    \begin{lstlisting}
    auto server_sock = @connect@(
        sock
        , reinterpret_cast<sockaddr*>(&addr)
        , sizeof(addr)
    );
    \end{lstlisting}}

    Zmienna \texttt{server\_sock} zawiera file descriptor wskazujący na socket,
    który umożliwia komunikację z serwerem.
\end{frame}

\begin{frame}
    \frametitle{Komunikacja}
    \framesubtitle{TCP/IP}

    Socket, który w ten sposób został stworzony reprezentuje \emph{połączenie}
    między serwerem, a klientem, które jest \emph{strumieniowe}.

    Oznacza to, że dane zapisane jednym wywołaniem \texttt{write(2)} na
    serwerze, nie muszą wcale być możliwe do odczytania jednym wywołaniem
    \texttt{read(2)} na kliencie, i \textit{vice versa} -- połączenie
    strumieniowe nie ma pojęcia ,,wiadomości''. Musi to być zaimplementowane w
    programie (np. definiując schemat danych, które są wysyłane -- protokół).
\end{frame}

\begin{frame}
    \frametitle{Opóźnienia i błędy}
    \framesubtitle{TCP/IP}

    Należy pamiętać, że komunikacja po sieci może być \textbf{wolna} i
    \textbf{zawodna}. Bardzo istotna jest poprawna obsługa możliwych błędów.

    \vspace{1em}

    Współcześnie używana pamięć masowa potrafi przekazywać dane z prędkościami
    sięgającymi kilku GB/s (\emph{gigabajtów} na sekundę). Połączenia sieciowe
    rzadko osiągają prędkość 1 Gb/s (\emph{gigabitu} na sekundę).

    \vspace{1em}

    Co więcej, sieć to nie połączenie SATA lub PCIe, tylko kable, kabelki i fale
    radiowe. Ta infrastruktura może zawieść - fale mogą być zagłuszone, a kable
    przerwane, co może wymagać użycia wolniejszej ścieżki lub doprowadzić do
    zerwania połączenia.
\end{frame}

\begin{frame}
    \frametitle{Zadanie: serwer echo}
    \label{lecture_exercise_3}

    Napisz serwer, który będzie oczekiwać na połączenia od klientów, odczytywać
    od nich jakieś dane (zakończone znakiem nowej linii), oraz odsyłać im te
    dane z powrotem. Pamiętaj o obsłużeniu sytuacji, w której klient się
    rozłącza!

    Kod źródłowy: \texttt{s06-echo-server.cpp}
\end{frame}

\begin{frame}
    \frametitle{Zadanie: serwer plików (odczyt)}
    \label{lecture_exercise_4}

    Napisz serwer, który będzie oczekiwać na połączenia od klientów, odczytywać
    od nich dane (zakończone znakiem nowej linii) reprezentujące ścieżki do
    plików.

    Po odebraniu ścieżki serwer powinien odczytać plik, i wysłać do klienta
    ścieżkę, znak nowej linii, oraz zawartośc pliku. Jeśli plik nie istnieje
    serwer powinien odesłać smutną buźkę: \texttt{:-(}

    Kod źródłowy: \texttt{s06-file-server.cpp}
\end{frame}

\begin{frame}
    \frametitle{Zadanie: wymiana wiadomości}
    \label{lecture_exercise_5}

    Napisz serwer, który będzie umożliwiał klientom wysłanie wiadomości do
    serwera. Serwer powinien te wiadomości zapisywać w pamięci, i odpowiadać
    losową wiadomością (jedną z tych które otrzymał już od innych klientów).

    Kod źródłowy: \texttt{s06-random-note-server.cpp}
\end{frame}

\begin{frame}
    \frametitle{Zadanie: client}
    \label{lecture_exercise_6}

    Napisz program, który będzie umożliwiał połączenie się z wybranym serwerem,
    wysyłanie do niego linijek tekstu, i odbieranie od niego danych. Użyj tego
    klienta do komunikacji z serwerami z poprzednich zadań.

    Kod źródłowy: \texttt{s06-client.cpp}
\end{frame}

\section{Async I/O}

\begin{frame}
    \frametitle{API}

    \begin{enumerate}
        \item \texttt{select(2)}
        \item \texttt{poll(2)}
        \item \texttt{epoll(7)} (Linux)
        \item \texttt{aio(7)}
    \end{enumerate}

    Podane strony manuala zawierają przykładowy kod w języku C, który można
    przepisać, skompilować, i sprawdzić jak działa.

    \vspace{1em}
    {\tiny
    System Linux oferuje jeszcze interfejs \texttt{io\_uring}, który jest na
    tyle nowy, że nie ma jeszcze strony w manualu -- informacji o nim trzeba
    szukać w Internecie.}
\end{frame}

\begin{frame}
    \frametitle{Zastosowanie}

    I/O zajmuje czas i może zablokować nasz wątek. Jeśli istnieje potrzeba
    uniknięcia takiej sytuacji, to można poradzić sobie z tym na kilka sposobów:

    \begin{enumerate}
        \item utworzyć wątek dedykowany operacjom I/O, który może być blokowany
        \item użyć \emph{nonblocking I/O}, czyli wywołań z gwarancją nieblokowania
            wątku (ten styl wiąże się z dodatkowymi wyzwaniami)
        \item użyć \emph{asynchronous I/O}, czyli wywołań, które powodują
            wykonanie operacji I/O ,,w tle'' przez system operacyjny
    \end{enumerate}
\end{frame}

\begin{frame}
    \frametitle{Przykłady}

    Wykonywanie kilku operacji I/O naraz, lub operacji I/O i czegoś jeszcze,
    jest bardzo częstą sytuacją - np.:

    \begin{enumerate}
        \item serwer obsługuje kilka klientów naraz
        \item program odczytuje kopiuje pliki i drukuje pasek postępu
        \item gra wczytuje poziom i pokazuje użytkownikowi ,,tip of the day''
        \item czat oczytuje wiadomości, ale jednocześnie drukuje powiadomienia
            ,,ktoś pisze...''
    \end{enumerate}
\end{frame}

\section{Podsumowanie}

\begin{frame}
    \frametitle{Co nowego?}
    \frametitle{Podsumowanie}

    Student powinien umieć:

    \begin{enumerate}
        \item prowadzić operacje I/O na plikach
        \item prowadzić operacje I/O z wykorzystaniem połączenia sieciowego
            używając protokołu TCP/IP
        \item znać możliwości i przykłady zastosowania API dostarczanego przez
            standard POSIX dla asychronicznych i nieblokujących operacji I/O
    \end{enumerate}
\end{frame}

\begin{frame}
    \frametitle{Zadania}
    \framesubtitle{Podsumowanie}

    Zadania znajdują się na slajdach
    \ref{lecture_exercise_0},
    \ref{lecture_exercise_1},
    \ref{lecture_exercise_2},
    \ref{lecture_exercise_3},
    \ref{lecture_exercise_4},
    \ref{lecture_exercise_5},
    \ref{lecture_exercise_6}.
\end{frame}

\end{document}
