\documentclass[aspectratio=169]{beamer}

\usepackage[utf8]{inputenc}
\usepackage{textcomp}
\usepackage[polish]{babel}
\usepackage{amsthm}
\usepackage{graphicx}
\usepackage[T1]{fontenc}
\usepackage{scrextend}
\usepackage{hyperref}
\usepackage{xcolor}
\usepackage{geometry}
\usepackage{listings}

\usetheme{-bjeldbak/beamerthemebjeldbak}

\definecolor{xbarcolor}{HTML}{0f2f0f}
\setbeamercolor{lower separation line head}{bg=xbarcolor} 
\setbeamercolor{lower separation line foot}{bg=xbarcolor} 

\title{Podstawy programownia (w języku C++)}
\subtitle{Sprawy organizacyjne, plan zajęć, narzędzia}
\author{Marek Marecki}
\institute{Polsko-Japońska Akademia Technik Komputerowych}

\lstset{basicstyle=\ttfamily\color{black},
columns=fixed,
escapeinside={\%*}{*)},
inputencoding=utf8,
extendedchars=true,
moredelim=**[is][\color{red}]{@}{@}}

\begin{document}

{%
    \setbeamertemplate{headline}{}
    \frame{\titlepage}
}

\section{Sprawy organizacyjne}

\begin{frame}
    \frametitle{Kontakt z prowadzącym}

    MS Teams lub email ({\tt marecki@pjwstk.edu.pl}).\\
    Lepiej email.
\end{frame}

\begin{frame}
    \frametitle{Zwolnienia z zajęć}

    Negocjowane indywidualnie (przez email - w tej kwestii komunikacja musi być
    oficjalna).
    Wymagana udokumentowana znajomość języka C++.

    \vspace{1em}

    Zwolnienie z zajęć nie zwalnia z napisania projektu (trzeba mieć dla uczelni
    jakiś ślad).
\end{frame}

\begin{frame}
    \frametitle{Zaliczenie zajęć}

    \begin{enumerate}
        \item aktywność w trakcie semestru -- pisanie programów i oddawanie ich
            na czas
        \item projekt oddany na koniec semestru (temat ustalany indywidualnie)
    \end{enumerate}

    \vspace{1em}

    Obie części warte są po 50\% oceny z ćwiczeń. Obie części muszą być
    zaliczone na co najmniej 3.
\end{frame}

\begin{frame}
    \frametitle{Śledzenie zmian w kodzie}

    Zmiany w kodzie student rejestruje używając narzędzia Git.\\
    Wymagane jest konto w serwisie Github: s\emph{indeks}, np. s12345. Powinno
    być założone na uczelniany adres email.
\end{frame}

\section{Plan zajęć}

\begin{frame}
    \frametitle{Płytka woda}

    Pierwsza połowa semestru obejmuje wprowadzenie do programowania jako
    takiego, przedstawienie fundamentów teoretcznych języków programowania, i
    zapoznanie się z podstawami języka C++.

    \vspace{1em}

    Zajęcia polegają na poznawaniu kolejnych konstrukcji języka C++ i pisaniu
    programów mających na celu ogólne obycie się studenta z programowaniem i
    narzędziami rzemiosła.
\end{frame}

\begin{frame}
    \frametitle{Głębkoa woda}

    Druga połowa semestru jest wykorzystana na projekt: napisanie przez studenta
    pojedynczego, rozbudowanego programu. Tematy projektów ustalane są
    indywidualnie.

    \vspace{1em}

    Zajęcia polegają na rozwiązywaniu prawdziwych problemów napotkanych przez
    studentów i poznawanie bardziej zaawansowanych aspektów języka.
\end{frame}

\begin{frame}
    \frametitle{Dżentelmeńska umowa}

    Na zajęcia $n$ student przychodzi mając opanowany materiał z zajęć $n-1$.

    \vspace{1em}

    Indywidualne wątpliwości i problemy związane z przedmiotem rozwiązywane są
    wraz z prowadzącym pomiędzy zajęciami.
\end{frame}

\section{Narzędzia}

\begin{frame}
    \frametitle{System operacyjny}

    Zajęcia prowadzone są na systemie Linux (dowolna dystrybucja).

    {\tiny Maszyna wirtualna do zajęć:
    \url{/srv/http/marecki.me/edu.marecki.me/Podstawy_programowania.ova}}
\end{frame}

\begin{frame}
    \frametitle{Kompilator}
    Wymagane kompilatory to GCC lub Clang w najnowszej dostępnej wersji.

    Kod źródłowy musi być formatowany narzędziem {\tt clang-format}; reguły
    formatowania udostępnia prowadzący zajęcia (do negocjacji).

    {\tiny Link do reguł:
    {\tt
    \url{https://git.sr.ht/~maelkum/education-introduction-to-programming-cxx/blob/master/.clang-format}}}
\end{frame}

\begin{frame}
    \frametitle{Flagi kompilacji}

    Kod źródłowy jest kompilowany użwając standardu C++14.

    Kod źródłowy musi kompilować się bez żadnych ostrzeżeń
    (\emph{warnings}).

    \vspace{1em}

    Wymagane flagi, których należy używać:\\
    {\tt
    -Wall -Wextra -Werror -Wfatal-errors -pedantic -g
    -Wzero-as-null-pointer-constant
    -Wuseless-cast
    -Wold-style-cast
    -Wswitch-default
    -Wswitch-enum
    -Wconversion
    -Wsign-conversion
    }
\end{frame}

\begin{frame}
    \frametitle{Szablon repozytorium}

    Polecam sklonować repozytorim przedmiotu. Są tam przykładowe programy,
    którymi można się posiłkować i plik {\tt Makefile}, który automatyzuje
    proces budowania programów.

    \vspace{1em}

    \url{https://git.sr.ht/~maelkum/education-introduction-to-programming-cxx}

    \vspace{2em}

    {\small Repozytorium zawiera też dodatkowe materiały dotyczące narzędzi
    używanych na zajęciach.}
\end{frame}

\end{document}
